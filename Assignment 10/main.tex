%%%%%%%%%%%%%%%%%%%%%%%%%%%%%%%%%%%%%%%%%%%%%%%%%%%%%%%%%%%%%%%
%
% Welcome to Overleaf --- just edit your LaTeX on the left,
% and we'll compile it for you on the right. If you open the
% 'Share' menu, you can invite other users to edit at the same
% time. See www.overleaf.com/learn for more info. Enjoy!
%
%%%%%%%%%%%%%%%%%%%%%%%%%%%%%%%%%%%%%%%%%%%%%%%%%%%%%%%%%%%%%%%


% Inbuilt themes in beamer
\documentclass{beamer}

% Theme choice:
\usetheme{CambridgeUS}

\setbeamertemplate{caption}[numbered]{}

\usepackage{enumitem}
\usepackage{tfrupee}
\usepackage{amsmath}
\usepackage{amssymb}
\usepackage{gensymb}
\usepackage{graphicx}
\usepackage{txfonts}

\def\inputGnumericTable{}

\usepackage[latin1]{inputenc}                                 
\usepackage{color}                                            
\usepackage{array}                                            
\usepackage{longtable}                                        
\usepackage{calc}                                             
\usepackage{multirow}                                         
\usepackage{hhline}                                           
\usepackage{ifthen}
\usepackage{caption} 
\captionsetup[table]{skip=3pt}  
\providecommand{\pr}[1]{\ensuremath{\Pr\left(#1\right)}}
\providecommand{\cbrak}[1]{\ensuremath{\left\{#1\right\}}}
\renewcommand{\thefigure}{\arabic{table}}
\renewcommand{\thetable}{\arabic{table}}

% Title page details: 
\title{AI1110: Assignment 10} 
\author{Aryan Sharan Reddy\\BT21BTECH11002}
\date{\today}
%\logo{\large \LaTeX{}}

\providecommand{\pr}[1]{\ensuremath{\Pr\left(#1\right)}}

\begin{document}

% Title page frame
\begin{frame}
    \titlepage 
\end{frame}

% Remove logo from the next slides
%\logo{}


% Outline frame
\begin{frame}{Outline}
    \tableofcontents
\end{frame}


% Lists frame
\section{Question}
\begin{frame}{Question}
Show that 
    
\begin{center}

R_{xy}(\lambda)= \lim_{T \to \infty} \dfrac{1}{2T} \int_{-T}^{T} x(t+ \lambda)y(t)dt\\

\end{center}
iff
\begin{center}
    \lim_{T \to \infty} \dfrac{1}{2T} \int_{-2T}^{2T} (1- \dfrac{|\tau|}{2T}) E\{x(t+ \lambda + \tau)y(t + \tau)x(t+ \lambda)y(t)\}d\tau= R^2_{xy}(\lambda) 
\end{center}
\end{frame}

\section{Solution}
\begin{frame}{Solution}

If $z(t)=x(t+\lambda)y(t)$, then

\begin{equation}
    C_{zz}(\tau) = E\{x(t+ \lambda + \tau)y(t + \tau)x(t+ \lambda)y(t)\}d\tau - R^2_{xy}(\lambda)
\end{equation}
%Since, z(t) is stationary, we have it's autocovariance as
Now we know that a process x(t) with autocovariance $C(\tau)$ is mean-ergodic iff
\begin{align}
    C_{zz}(\tau) &= \dfrac{1}{2T} \int_{-2T}^{2T} C(\tau)(1-\dfrac{|\tau|}{2T})d\tau\\
    &= \dfrac{1}{T} \int_{0}^{2T} C(\tau)(1-\dfrac{|\tau|}{2T})d\tau \longrightarrow_{T \to \infty} 0
\end{align}
The given process is mean-ergodic

{\implies} R_{xy}(\lambda)= \lim_{T \to \infty} \dfrac{1}{2T} \int_{-T}^{T} x(t+ \lambda)y(t)dt

\end{frame}
\end{document}
