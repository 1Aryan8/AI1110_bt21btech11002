\let\negmedspace\undefined
\let\negthickspace\undefined
%\RequirePackage{amsmath}
\documentclass[journal,12pt,twocolumn]{IEEEtran}
%
\usepackage{setspace}
 \usepackage{gensymb}
%\doublespacing
 \usepackage{polynom}
%\singlespacing
%\usepackage{silence}
%Disable all warnings issued by latex starting with "You have..."
%\usepackage{graphicx}
\usepackage{amssymb}
%\usepackage{relsize}
\usepackage[cmex10]{amsmath}
%\usepackage{amsthm}
%\interdisplaylinepenalty=2500
%\savesymbol{iint}
%\usepackage{txfonts}
%\restoresymbol{TXF}{iint}
%\usepackage{wasysym}
\usepackage{amsthm}
%\usepackage{pifont}
%\usepackage{iithtlc}
% \usepackage{mathrsfs}
% \usepackage{txfonts}
 \usepackage{stfloats}
% \usepackage{steinmetz}
 \usepackage{bm}
% \usepackage{cite}
% \usepackage{cases}
% \usepackage{subfig}
%\usepackage{xtab}
\usepackage{longtable}
%\usepackage{multirow}
%\usepackage{algorithm}
%\usepackage{algpseudocode}
\usepackage{enumitem}
 \usepackage{mathtools}
 \usepackage{tikz}
% \usepackage{circuitikz}
% \usepackage{verbatim}
%\usepackage{tfrupee}
\usepackage[breaklinks=true]{hyperref}
%\usepackage{stmaryrd}
%\usepackage{tkz-euclide} % loads  TikZ and tkz-base
%\usetkzobj{all}
\usepackage{listings}
    \usepackage{color}                                            %%
    \usepackage{array}                                            %%
    \usepackage{longtable}                                        %%
    \usepackage{calc}                                             %%
    \usepackage{multirow}                                         %%
    \usepackage{hhline}                                           %%
    \usepackage{ifthen}                                           %%
  %optionally (for landscape tables embedded in another document): %%
    \usepackage{lscape}     
% \usepackage{multicol}
% \usepackage{chngcntr}
%\usepackage{enumerate}
\usepackage{tfrupee}

%\usepackage{wasysym}
%\newcounter{MYtempeqncnt}
\DeclareMathOperator*{\Res}{Res}
\DeclareMathOperator*{\equals}{=}
%\renewcommand{\baselinestretch}{2}
%\renewcommand\thesection{\arabic{section}}
%\renewcommand\thesubsection{\thesection.\arabic{subsection}}
%\renewcommand\thesubsubsection{\thesubsection.\arabic{subsubsection}}

%\renewcommand\thesectiondis{\arabic{section}}
%\renewcommand\thesubsectiondis{\thesectiondis.\arabic{subsection}}
%\renewcommand\thesubsubsectiondis{\thesubsectiondis.\arabic{subsubsection}}

% correct bad hyphenation here
\hyphenation{op-tical net-works semi-conduc-tor}
\def\inputGnumericTable{}                                 %%

\lstset{
%language=C,
frame=single, 
breaklines=true,
columns=fullflexible
}
%\lstset{
%language=tex,
%frame=single, 
%breaklines=true
%}


\title{Assignment 1 \\ \Large AI1110: Probability and Random Variables \\ \large Indian Institute of Technology Hyderabad}
\author{Aryan Sharan Reddy \\ \normalsize BT21BTECH11002 \\ \vspace*{20pt} \normalsize  29 March 2022 \\ \vspace*{20pt} \Large ICSE 2019 Class 12}


\begin{document}

%


\newtheorem{theorem}{Theorem}[section]
\newtheorem{problem}{Problem}
\newtheorem{proposition}{Proposition}[section]
\newtheorem{lemma}{Lemma}[section]
\newtheorem{corollary}[theorem]{Corollary}
\newtheorem{example}{Example}[section]
\newtheorem{definition}[problem]{Definition}
%\newtheorem{thm}{Theorem}[section] 
%\newtheorem{defn}[thm]{Definition}
%\newtheorem{algorithm}{Algorithm}[section]
%\newtheorem{cor}{Corollary}
\newcommand{\BEQA}{\begin{eqnarray}}
\newcommand{\EEQA}{\end{eqnarray}}
\newcommand{\define}{\stackrel{\triangle}{=}}
\newcommand*\circled[1]{\tikz[baseline=(char.base)]{
    \node[shape=circle,draw,inner sep=2pt] (char) {#1};}}
\bibliographystyle{IEEEtran}
%\bibliographystyle{ieeetr}
\providecommand{\mbf}{\mathbf}
\providecommand{\pr}[1]{\ensuremath{\Pr\left(#1\right)}}
\providecommand{\qfunc}[1]{\ensuremath{Q\left(#1\right)}}
\providecommand{\sbrak}[1]{\ensuremath{{}\left[#1\right]}}
\providecommand{\lsbrak}[1]{\ensuremath{{}\left[#1\right.}}
\providecommand{\rsbrak}[1]{\ensuremath{{}\left.#1\right]}}
\providecommand{\brak}[1]{\ensuremath{\left(#1\right)}}
\providecommand{\lbrak}[1]{\ensuremath{\left(#1\right.}}
\providecommand{\rbrak}[1]{\ensuremath{\left.#1\right)}}
\providecommand{\cbrak}[1]{\ensuremath{\left\{#1\right\}}}
\providecommand{\lcbrak}[1]{\ensuremath{\left\{#1\right.}}
\providecommand{\rcbrak}[1]{\ensuremath{\left.#1\right\}}}
\theoremstyle{remark}
\newtheorem{rem}{Remark}
\newcommand{\sgn}{\mathop{\mathrm{sgn}}}
\providecommand{\fourier}{\overset{\mathcal{F}}{ \rightleftharpoons}}
%\providecommand{\hilbert}{\overset{\mathcal{H}}{ \rightleftharpoons}}
\providecommand{\system}{\overset{\mathcal{H}}{ \longleftrightarrow}}
	%\newcommand{\solution}[2]{\textbf{Solution:}{#1}}
\newcommand{\solution}{\noindent \textbf{Solution: }}
\newcommand{\cosec}{\,\text{cosec}\,}
\providecommand{\dec}[2]{\ensuremath{\overset{#1}{\underset{#2}{\gtrless}}}}
\newcommand{\myvec}[1]{\ensuremath{\begin{pmatrix}#1\end{pmatrix}}}
\newcommand{\mydet}[1]{\ensuremath{\begin{vmatrix}#1\end{vmatrix}}}
%\numberwithin{equation}{section}
%\numberwithin{figure}{section}
%\numberwithin{table}{section}
%\numberwithin{equation}{subsection}
%\numberwithin{problem}{section}
%\numberwithin{definition}{section}
\makeatletter
\@addtoreset{figure}{problem}
\makeatother
\let\StandardTheFigure\thefigure
\let\vec\mathbf
%\renewcommand{\thefigure}{\theproblem.\arabic{figure}}
%\renewcommand{\thefigure}{\theproblem}
%\setlist[enumerate,1]{before=\renewcommand\theequation{\theenumi.\arabic{equation}}
%\counterwithin{equation}{enumi}
%\renewcommand{\theequation}{\arabic{subsection}.\arabic{equation}}
\def\putbox#1#2#3{\makebox[0in][l]{\makebox[#1][l]{}\raisebox{\baselineskip}[0in][0in]{\raisebox{#2}[0in][0in]{#3}}}}
     \def\rightbox#1{\makebox[0in][r]{#1}}
     \def\centbox#1{\makebox[0in]{#1}}
     \def\topbox#1{\raisebox{-\baselineskip}[0in][0in]{#1}}
     \def\midbox#1{\raisebox{-0.5\baselineskip}[0in][0in]{#1}}
\title{
	%\logo{
%Computational Approach to School Geometry
	% The title
		Assignment 2
%	}
}
\author{ Aryan Sharan Reddy (BT21BTECH11002)% <-this % stops a space
}
\graphicspath{{figures/}}
%\title{
%	\logo{Matrix Analysis through Octave}{\begin{center}\includegraphics[scale=.24]{tlc}\end{center}}{}{HAMDSP}
%}
% paper title
% can use linebreaks \\ within to get better formatting as desired
%\title{Matrix Analysis through Octave}
%
%
% author names and IEEE memberships
% note positions of commas and nonbreaking spaces ( ~ ) LaTeX will not break
% a structure at a ~ so this keeps an author's name from being broken across
% two lines.
% use \thanks{} to gain access to the first footnote area
% a separate \thanks must be used for each paragraph as LaTeX2e's \thanks
% was not built to handle multiple paragraphs
%
%\author{<-this % stops a space
%\thanks{}}
%}
% note the % following the last \IEEEmembership and also \thanks - 
% these prevent an unwanted space from occurring between the last author name
% and the end of the author line. i.e., if you had this:
% 
% \author{....lastname \thanks{...} \thanks{...} }
%                     ^------------^------------^----Do not want these spaces!
%
% a space would be appended to the last name and could cause every name on that
% line to be shifted left slightly. This is one of those "LaTeX things". For
% instance, "\textbf{A} \textbf{B}" will typeset as "A B" not "AB". To get
% "AB" then you have to do: "\textbf{A}\textbf{B}"
% \thanks is no different in this regard, so shield the last } of each \thanks
% that ends a line with a % and do not let a space in before the next \thanks.
% Spaces after \IEEEmembership other than the last one are OK (and needed) as
% you are supposed to have spaces between the names. For what it is worth,
% this is a minor point as most people would not even notice if the said evil
% space somehow managed to creep in.
%\WarningFilter{latex}{LaTeX Warning: You have requested, on input line 117, version}
% The paper headers
%\markboth{Journal of \LaTeX\ Class Files,~Vol.~6, No.~1, January~2007}%
%{Shell \MakeLowercase{\textit{et al.}}: Bare Demo of IEEEtran.cls for Journals}
% The only time the second header will appear is for the odd numbered pages
% after the title page when using the twoside option.
% 
% *** Note that you probably will NOT want to include the author's ***
% *** name in the headers of peer review papers.                   ***
% You can use \ifCLASSOPTIONpeerreview for conditional compilation here if
% you desire.
% If you want to put a publisher's ID mark on the page you can do it like
% this:
%\IEEEpubid{0000--0000/00\$00.00~\copyright~2007 IEEE}
% Remember, if you use this you must call \IEEEpubidadjcol in the second
% column for its text to clear the IEEEpubid mark.
% make the title area
\maketitle
%\renewcommand{\theequation}{\theenumi}
%\begin{abstract}
%%\boldmath
%In this letter, an algorithm for evaluating the exact analytical bit error rate  (BER)  for the piecewise linear (PL) combiner for  multiple relays is presented. Previous results were available only for upto three relays. The algorithm is unique in the sense that  the actual mathematical expressions, that are prohibitively large, need not be explicitly obtained. The diversity gain due to multiple relays is shown through plots of the analytical BER, well supported by simulations. 
%
%\end{abstract}
% IEEEtran.cls defaults to using nonbold math in the Abstract.
% This preserves the distinction between vectors and scalars. However,
% if the journal you are submitting to favors bold math in the abstract,
% then you can use LaTeX's standard command \boldmath at the very start
% of the abstract to achieve this. Many IEEE journals frown on math
% in the abstract anyway.
% Note that keywords are not normally used for peerreview papers.
%\begin{IEEEkeywords}
%Cooperative diversity, decode and forward, piecewise linear
%\end{IEEEkeywords}
% For peer review papers, you can put extra information on the cover
% page as needed:
% \ifCLASSOPTIONpeerreview
% \begin{center} \bfseries EDICS Category: 3-BBND \end{center}
% \fi
%
% For peerreview papers, this IEEEtran command inserts a page break and
% creates the second title. It will be ignored for other modes.
%\IEEEpeerreviewmaketitle

\begin{abstract}
This document contains the solution for Assignment 2 (ICSE Class 12 Maths 2019 Q.12(a))
\end{abstract}



%template ends here





%main text begins



%\begin{flushleft}
	% The title
	\maketitle
	
	% The question
	\textbf{Question 12(a)} 
	 The volume of a closed rectangular metal box with a square base is $4096$ cm$^3$. The cost of polishing the outer surface of the box is \rupee $4$ per cm$^2$. Find the dimensions of the box at the minimum cost of polishing it.
	 
	 % The solution
	\textbf{\underline{Solution.}} Let the volume of the closed rectangular metal box be V.
	
	Given that the volume of this box is 4096 cm$^3$.
	
	
	 \begin{equation}
	 \label{eq:1}{\implies}V = 4096 cm^3   
	 \end{equation} 
	 
	 Given that the box has a square base which means that the breadth and height of the box are equal. Let their value be $a$.
	 
	 And let the value of length be $b$.
	 
	 In this case, we have
	 
	 
	 \begin{equation}
	   \label{eq:2}V=a^2b
	 \end{equation}
	 
	 From \eqref{eq:1} and \eqref{eq:2}, we have
	 
	 \begin{equation}
	    a^2b=4096
	 \end{equation}
	 
	 Getting $b$ in terms of $a$, we get
	 
	 \begin{equation}
	    \label{eq:4} b = \dfrac{4096}{a^2}
	 \end{equation}
	 
	 In the second part of the question, it is given that the cost of polishing the outer surface of the box is {\rupee}4 per cm$^2$
	 
	 Let the total surface area of the box be S.
	 
	 %Total surface area of our metal box(if kept vertically) = surface area of the two square surfaces(top and bottom) + surface area of the 4 rectangular surfaces(walls).\\
	 \begin{equation}
	    S = 2a^2 + 4ab
	 \end{equation}
	 
	 From \eqref{eq:4}, we have
	 
	 \begin{align}
	  S &= 2a^2 + 4a(\dfrac{4096}{a^2})\\
	    &= 2a^2 + 4(\dfrac{4096}{a})\\
	    &= 2a^2 + \dfrac{16384}{a}\\
	   \label{eq:9} \therefore S &= 2a^2 + \dfrac{16384}{a}
	    \end{align}
	    
	    Now let $S = y = f(a)$
	    
	    We need to find the a at where f(a) is minimum by using gradient descent method.
	    
Let us find $\nabla f(a)$
\begin{align}
    \frac{dy}{da} &= \frac{d}{da}\brak{2a^2 + \dfrac{16384}{a}} \\
   \implies f^{\prime}(a) &= 4a-\dfrac{16384}{a^2}
\end{align}

We will be able to find the corresponding a value of the minimum of $f(a)$ by iterating the following equation till $\brak{f^{\prime}\brak{a_{k-1}}}$ approaches zero.

\begin{align}
a_{k} = a_{k-1} - \brak{\alpha\times f^{\prime}\brak{a_{k-1}}} 
\end{align}
where $a_{k-1}$ is initial assumed value/ previous obtained value

$a_k$ is updated assumed value

$\alpha$ represents the step size we are taking according to the slope $\brak{f^{\prime}\brak{a_{k-1}}}$

At first, let us randomly choose $a_{k-1}$ as $4$. Then, $f^{\prime}\brak{4} = -1008 $.

Since the slope is too far from zero and for manual purpose, we can take large step size. Hence let us choose $\alpha$ as $0.125$ 

Lets go through couple of iterations


\begin{align}
    a_k &= 4     - 0.125\times\brak{-1008}\\
    &= 130 \\
    a_k &= 130 - 0.125\times\brak{519.03053}\\
    &= 65.12118 \\
    a_k &= 65.12118 - 0.125\times\brak{256.62127} \\
    &= 33.04353\\
    a_k &= 33.04353 - 0.125\times\brak{117.16874}\\
    &= 18.39744 \\
    a_k &= 18.39744 - 0.125\times\brak{25.1831}\\
    &= 15.24956 \\
    a_k &= 15.24956 - 0.125\times\brak{-9.4557}\\
    &= 16.43152 \\
    a_k &= 16.43152 - 0.125\times\brak{5.04345}\\
    &= 15.80109 \\
    a_k &= 15.80109 - 0.125\times\brak{-2.41709}\\
    &= 16.10322 \\
    a_k &= 16.10322 - 0.125\times\brak{1.23072}\\
    &= 15.94938
\end{align}

Clearly, we observe that $a_k$ is tending to 16 from both the left hand side as well as the right hand side. Hence the possible whole number at where the minimum of $f(a)$ exists is $a=16$.
%\textbf{Note}: For solving the minimum using gradient descent method with5 algorithms, we can iterate through a lot of times to obtain the more precise value and we can take small step size too.\newline

	  %For minimum cost we must have minimum surface area.
	    
	  %Now, for minimum value of surface area, we must have
	  %\begin{align}
	  %\label{eq:10}\dfrac{dS}{da}&=0\\
	  %\dfrac{dS}{da} &= 4a-\dfrac{16384}{a^2}
	  %\end{align}
	  
	  %From \eqref{eq:10}, we have
	  
	  %\begin{align}4a-\dfrac{16384}{a^2} &= 0\\
	   %\implies4a &= \dfrac{16384}{a^2}\\
	  %\implies a^3 &= 4096\\    
	 %\implies a &= 16
	 %\end{align}
	 
	 %Taking the second derivative of S, we get
	 
	 %\begin{equation}
	 %\label{eq:16}\dfrac{d^2S}{da^2} = 4 + \dfrac{32768}{a^3}
	 %\end{equation}\\
	 %Clearly \eqref{eq:16} is positive for all positive values of a.
	 %S has a minimum at a = 16
	 
	  Put $a = 16$ in \eqref{eq:9},
	  
	  %\pagebreak
	\begin{align}
	S_m &= 2(16)^2 + \dfrac{16384}{16}\\
	 &= 2(256) + 1024\\
	&= 512 + 1024\\
	&= 1536\\
	 \label{eq:21}\therefore S_m &= 1536 cm^2
	 \end{align}
	 
	 Let the cost per unit area be c which is equal to \rupee$4 per cm{^2}$
	 
	  Let the minimum cost of polishing the metal box be $C_m$
	  \begin{equation}
	  \implies C_m = c \times S_m
	  \end{equation}
	  
	 From \eqref{eq:21}, we have
	 
	  \begin{align} C_m &= 4\times1536\\
	   &=3072
	   \end{align}
	   \text{\fbox{{\therefore} $The minimum cost of polishing the metal box is  {\rupee 3072}$}}
	   \begin{table}[h] 
\caption{\textbf{Design Table}}
\label{Table}
\input{table.tex}
\end{table}
	   
%\end{flushleft}	 
\end{document}
