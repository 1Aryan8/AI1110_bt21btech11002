%%%%%%%%%%%%%%%%%%%%%%%%%%%%%%%%%%%%%%%%%%%%%%%%%%%%%%%%%%%%%%%
%
% Welcome to Overleaf --- just edit your LaTeX on the left,
% and we'll compile it for you on the right. If you open the
% 'Share' menu, you can invite other users to edit at the same
% time. See www.overleaf.com/learn for more info. Enjoy!
%
%%%%%%%%%%%%%%%%%%%%%%%%%%%%%%%%%%%%%%%%%%%%%%%%%%%%%%%%%%%%%%%


% Inbuilt themes in beamer
\documentclass{beamer}

% Theme choice:
\usetheme{CambridgeUS}

\setbeamertemplate{caption}[numbered]{}

\usepackage{enumitem}
\usepackage{tfrupee}
\usepackage{amsmath}
\usepackage{amssymb}
\usepackage{gensymb}
\usepackage{graphicx}
\usepackage{txfonts}

\def\inputGnumericTable{}

\usepackage[latin1]{inputenc}                                 
\usepackage{color}                                            
\usepackage{array}                                            
\usepackage{longtable}                                        
\usepackage{calc}                                             
\usepackage{multirow}                                         
\usepackage{hhline}                                           
\usepackage{ifthen}
\usepackage{caption} 
\captionsetup[table]{skip=3pt}  
\providecommand{\pr}[1]{\ensuremath{\Pr\left(#1\right)}}
\providecommand{\cbrak}[1]{\ensuremath{\left\{#1\right\}}}
\renewcommand{\thefigure}{\arabic{table}}
\renewcommand{\thetable}{\arabic{table}}

% Title page details: 
\title{AI1110: Assignment 5} 
\author{Aryan Sharan Reddy\\BT21BTECH11002}
\date{\today}
%\logo{\large \LaTeX{}}

\providecommand{\pr}[1]{\ensuremath{\Pr\left(#1\right)}}

\begin{document}

% Title page frame
\begin{frame}
    \titlepage 
\end{frame}

% Remove logo from the next slides
%\logo{}


% Outline frame
\begin{frame}{Outline}
    \tableofcontents
\end{frame}


% Lists frame
\section{Question}
\begin{frame}{Question}
  Consider the following three events: (a) At least 1 six is obtained when six dice are rolled, (b) at least 2 sixes are obtained when 12 dice are rolled, and (c) at least 3 sixes are obtained when 18 dice are rolled. Which of these events is more likely?  
\end{frame}

\section{Solution (a)}
\begin{frame}{Solution (a)}
   Possible outcomes when a die is rolled once are:
   \begin{center}
      $ \{1, 2, 3, 4, 5, 6\}$
   \end{center}
   In the first event, six dice are rolled.
   
   Let $X=i$ denote the event where six occurs i times on the dice $i\in \cbrak{0,1, 2,..., n}$ 
   
    Let $n = 6$ in this case
   
   Let the probability of obtaining at least one six be P_{n}($X \geqslant 1$) where $n = 6$
   
  \begin{align}
    P_{6}(X \geqslant 1)&= 1-P_{6}(X = 0)\\
    &= 1- ^6C_0 \brak{(1-p)}^{0}p^{6-0}\\
    &= 1- ^6C_0\brak(\dfrac{1}{6})^{0}(\dfrac{5}{6})^{6}\\
    &=1- (\dfrac{5}{6})^{6} = 0.66511
   \end{align}
   
\end{frame}

\section{Solution (b)}
\begin{frame}{Solution (b)}
    In the second event, twelve dice are rolled.
    
   Now we have $n=12$
   
   
   %Let $X=i$ denote the event where six occurs i times on the dice $i\in \cbrak{0,1, 2,..., n}$ 
   
   Let the probability of obtaining at least two sixes be P_{n}($X \geqslant 2$) where $n = 12$
   
   \begin{align}
   P_{12}(X \geqslant 2)&= 1-P_{12}(X \leq 1)\\
    &= 1-P_{12}(X = 0)-P_{12}(X = 1)\\
    %&= 1- ^{12}C_0 \brak{(1-p)}^{0}p^{12-0}-^{12}C_1\brak{(1-p)}^{1}p^{12-1}\\
    &=1- ^{12}C_0\brak{(\dfrac{1}{6})^{0}(\dfrac{5}{6})^{12}}-^{12}C_1\brak{(\dfrac{1}{6}})^{1}(\dfrac{5}{6})^{11} \\
    &=1- (\dfrac{5}{6})^{11}(\dfrac{17}{6}) = 0.61867
   \end{align}
   
\end{frame}

\section{Solution (c)}
\begin{frame}{Solution (c)}
    In the second event, eighteen dice are rolled.
    
   Now we have $n=18$
   
   
   %Let $X=i$ denote the event where six occurs i times on the dice $i\in \cbrak{0,1, 2,..., n}$ 
   
   Let the probability of obtaining at least two sixes be P_{n}($X \geqslant 3$) where $n = 18$
   
   \begin{align}
   P_{18}(X \geqslant 3)&= 1-P_{18}(X \leq 2)\\
    &= 1-P_{18}(X = 0)-P_{18}(X = 1)-P_{18}(X = 2)\\
    %&= 1- ^{12}C_0 \brak{(1-p)}^{0}p^{12-0}-^{12}C_1\brak{(1-p)}^{1}p^{12-1}\\
    &=1- ^{18}C_0\brak{(\dfrac{1}{6})^{0}(\dfrac{5}{6})^{18}}-^{18}C_1\brak{(\dfrac{1}{6}})^{1}(\dfrac{5}{6})^{17}-^{18}C_2\brak{(\dfrac{1}{6}})^{2}(\dfrac{5}{6})^{16} \\
    &=1- (\dfrac{5}{6})^{18}-18(\dfrac{1}{6})(\dfrac{5}{6})^{17}-153(\dfrac{1}){36}(\dfrac{5}{6})^{16}= 0.59078
   \end{align}

\end{frame}

\section{Conclusion}
\begin{frame}{Conclusion}
Clearly,

P_{6}(X \geqslant 1) > P_{12}(X \geqslant 2) > P_{18}(X \geqslant 3)

 From the above inequality, we conclude that the getting at least one six when the die is rolled six times is the most likely.

\end{frame}
\end{document}
